\chapter{VI}
\label{cha:VI}

VI es un editor de textos pequeño y versátil. Viene instalado en todos los
sistemas UNIX por lo que nos interesa conocerlo y saber utilizarlo.

Para editar un fichero utilizaremos:

\begin{lstlisting}
vi FICHERO
\end{lstlisting}

Existe una versión mejorada llamada VIM (\emph{Vi IMproved}) que habitualmente
está instalado. Esta versión incluye un modo no compatible (con VI) que
incluye más comandos y opciones que podéis encontrar en
\url{http://www.vim.org/}.

En este capítulo utilizaremos los comandos compatibles con VIM, que en su gran
mayoría están cubiertos en VI. En cualquier caso la recomendación es instalar
VIM y ejecutar \emph{vimtutor}.

\section{Modos} % (fold)
\label{sec:Modos}

En VI existen principalmente dos modos: el modo edición y el modo comando. La
principal diferencia es que en modo edición nuestras pulsaciones se escriben en
el fichero que estamos editando mientras que en modo comando son instrucciones
que el propio VI interpreta.

Al arrancar VI entramos en modo comando.

\subsection{Cambio entre modos} % (fold)
\label{sub:Cambio entre modos}

Para pasar del modo comando al modo edición tenemos que utilizar cualquiera de
los comandos de edición. Es decir, comandos de añadir texto, insertar texto,
cambiar texto, abrir línea de texto, etc\...

Para pasar del modo edición al modo comando tenemos que pulsar la tecla
\emph{Escape}.

% subsection Cambio entre modos (end)

% section Modos (end)

\section{Movimiento dentro de VI} % (fold)
\label{sec:Movimiento dentro de VI}

Dentro del modo comando podemos desplazarnos dentro del fichero:

\begin{description}
    \item[h] una letra hacia la izquierda
    \item[k] una letra hacia arriba
    \item[l] una letra hacia la derecha
    \item[j] una letra hacia abajo
    \item[w] comienzo de la siguente palabra
    \item[e] final de la palabra actual
    \item[b] comienzo de la palabra anterior
\end{description}

Los desplazamientos se pueden encadenar o repetir o bien de forma manual,
pulsando varias veces, o bien utilizando un multiplicador antes del comando.
Por ejemplo para bajar tres líneas podemos utilizar \textbf{3j}.

% section Movimiento dentro de VI (end)

\section{Buscar} % (fold)
\label{sec:Buscar}

Para buscar disponemos de dos comandos desde el modo comando:

\begin{description}
    \item[/] que busca desde la posición actual hacia abajo
    \item[?] que busca desde la posición actual hacia arriba
\end{description}

Podemos avanzar a la siguiente coincidencia con el comando \textbf{n}, en la
misma dirección, o bien con el comando \textbf{N}, en dirección contraria.

% section Buscar (end)

\section{Comandos de edición de texto} % (fold)
\label{sec:Comandos de edicion de texto}

Todos estos comandos se utilizan desde el modo de comandos y automáticamente
nos pasan al modo de edición. Es decir, el primer comando da la orden y a
continuación pasamos inmediatamente a editar el fichero.

\begin{description}
    \item[i] insertar a partir de la posición actual
    \item[a] añadir, insertar a partir de la siguiente posición de la actual
    \item[c + Movimiento] reemplaza el texto desde la posición actual hasta el
    fin del movimiento.
    \item[o] abrir una línea por debajo de la posición actual e insertar 
\end{description}

Vamos a ver con más detalle el comando \textbf{c}. Si queremos reemplazar la
palabra actual, nos colaremos en el comienzo de la palabra y ejecutaremos
\textbf{cw} que automáticamente pasará al modo edición substituyendo la
palabra.

De esta forma muchos comandos pueden afectar un rango de texto, desde la
posición actual hasta el desplazamiento que elijamos. Pudiendo encadenar varios
desplazamientos, por ejemplo reemplazar tres palabras con \textbf{c3w}.

La versión en mayúsculas de estos comandos realiza acciones similares:

\begin{description}
    \item[I] insertar en el comienzo de la línea actual
    \item[A] insertar al final de la línea actual
    \item[O] abrir una línea por encima de la posición actual e insertar
\end{description}

% section Comandos de edición de texto (end)

\section{Copia, cortar y pegar} % (fold)
\label{sec:Copia, cortar y pegar}

Desde el modo comando podemos realizar las operaciones típicas de cortar y
pegar un texto. Con estos comandos utilizaremos un desplazamiento relativo.

\begin{description}
    \item[d + Movimiento] elimina, cortando
    \item[dd] elimina, cortando, la línea actual
    \item[y + Movimiento] copia
    \item[yy] copia la línea actual
    \item[p] pega
\end{description}
% section Copia, cortar y pegar (end)

\section{Deshacer y rehacer} % (fold)
\label{sec:Deshacer y rehacer}

Tenemos dos comandos:

\begin{description}
    \item[u] deshace el último cambio realizado
    \item[Control + r] rehace el último deshacer
\end{description}

% section Deshacer y rehacer (end)

\section{Guardar y salir} % (fold)
\label{sec:Guardar y salir}

Existen múltiples comandos para guardar y salir de VI:

\begin{description}
    \item[:w] guarda el fichero
    \item[:e!] vuelve a abrir el fichero actual, perdiendo los cambios
    realizados
    \item[ZZ] guarda y sale
    \item[:q] sale
    \item[:q!] sale ignorando los cambios realizados
\end{description}

% section Guardar y salir (end)
