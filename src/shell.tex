\chapter{Explorar las herramientas de la línea de comandos}
\label{cha:Shell}

A diferencia de otros sistemas operativos, Linux tiene como herencia a Unix,
 que en su tiempo era un SO basado en texto.
Aún todavía, con las interfaces gráficas muy desarrolladas, los comandos de texto
son muy potentes en Linux, y mucho mas configurables que una aplicacion con interfaz.

Es por eso que este primer capitulo va dirigido a te familiarices con las consolas (shells)
de Linux, que no son otra cosa que intérpretes de comandos.


\section*{Conocer los fundamentos de la linea de comandos}


\subsection*{Explorar todas las opciones de la consola de Linux}
		 	

\subsection*{Utilizar una consola}



\subsubsection*{Iniciar una consola}
			 	

\subsubsection*{Uso de comandos internos y externos}
			 	

\subsubsection*{Algunos trucos de la consola de comandos}


		



\subsection*{Explorar la configuracion de la consola}
			


\subsection*{Uso de las variables de entorno}
			

\subsection*{Obtener ayuda}




