\chapter{La linea de comandos}
\label{cha:Shell}


A diferencia de otros sistemas operativos, Linux tiene como herencia a Unix,
 que en su tiempo era un SO basado en texto.
Aún todavía, con las interfaces gráficas muy desarrolladas, los comandos de texto
son muy potentes en Linux, y mucho mas configurables que una aplicacion con interfaz.

Es por eso que este primer capitulo va dirigido a te familiarices con las consolas (shells)
de Linux, que no son otra cosa que intérpretes de comandos.


\section*{Conocer los fundamentos de la linea de comandos}

Antes de hablar de redirecciones, pipes y filtros, tienes que saber cómo utilizar una 
consola de Linux.

Vamos allá.

\subsection*{Explorar todas las opciones de la consola de Linux}

Como en casi todo, las consolas de Linux son muy variadas, algunas entre si se parecen mucho, 
y otras no se parecen tanto. Aqui vamos a enseñarte las más conocidas y usadas por las distintas 
distribuciones.

\begin{tabbing}
0123\=456\=789\=\kill		 	

\> -bash: La Bourne Again Shell GNU (bash) es la consola más usada hoy\\ 
\> en día en las distribuciones Linux, está basada en la consola Bourne, \\
\> al ser la más usada va a ser la que veamos en este curso, y, por lo tanto,\\
\> la que vas a aprender.\\

\> -bsh: Es la consola padre de bash, se llama Bourne, actualmente el comando\\
\> bsh es un enlace simbólico a la consola bash.\\



\end{tabbing}


\subsection*{Utilizar una consola}



\subsubsection*{Iniciar una consola}
			 	

\subsubsection*{Uso de comandos internos y externos}
			 	

\subsubsection*{Algunos trucos de la consola de comandos}


		



\subsection*{Explorar la configuracion de la consola}
			


\subsection*{Uso de las variables de entorno}
			

\subsection*{Obtener ayuda}



